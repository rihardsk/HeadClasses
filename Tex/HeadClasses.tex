\documentclass{llncs}
\usepackage{graphicx}
\usepackage{float}
\usepackage{enumerate}
\usepackage{amssymb,amsmath}
\usepackage{enumitem}
\usepackage{verbatim}

\newcommand{\norm}[1]{\left\| #1 \right\|}

\begin{document}

\title{On the Classes of Finite Ultrametric automata}


\author{
Rihards Kri\v slauks}
\institute{University of Latvia Faculty of computing,\\ Rai\c na bulv\= aris 19, Riga, LV-1459, Latvia
}

\maketitle

\begin{abstract}
The work explores the language classes that arise with respect to the head count of a finite automaton. The results for deterministic non-deterministic and probabilistic automata are explored and similar results are proved for two-way ultrametric automata, which are viewed as a generalization of ultrametric finite automata that have just recently been introduced by ~\citet{Freivalds2012}. For the one-way setting it is shown that ultrametric one-head finite automata are more powerful than deterministic and non-deterministic automata for certain languages. Definitions for ultrametric Turing machines and ultrametric multi-register machines are introduced as a tool for proving the results. An interesting result regarding multi-tape automata is shown as well.
\end{abstract} 



\section{Introduction} 
