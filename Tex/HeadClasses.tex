\documentclass{llncs}
\usepackage{graphicx}
\usepackage{float}
\usepackage{enumerate}
\usepackage{amssymb,amsmath}
\usepackage{enumitem}
\usepackage{verbatim}

\newcommand{\norm}[1]{\left\| #1 \right\|}

\begin{document}

\title{On the Classes of Finite Ultrametric automata}


\author{
Rihards Kri\v slauks}
\institute{University of Latvia Faculty of computing,\\ Rai\c na bulv\= aris 19, Riga, LV-1459, Latvia
}

\maketitle

\begin{abstract}
The work explores the language classes that arise with respect to the head count of a finite automaton. The results for deterministic non-deterministic and probabilistic automata are explored and similar results are proved for two-way ultrametric automata, which are viewed as a generalization of ultrametric finite automata that have just recently been introduced by ~\citet{Freivalds2012}. For the one-way setting it is shown that ultrametric one-head finite automata are more powerful than deterministic and non-deterministic automata for certain languages. Definitions for ultrametric Turing machines and ultrametric multi-register machines are introduced as a tool for proving the results. An interesting result regarding multi-tape automata is shown as well.
\end{abstract}



\section{Introduction} 
Ultrametric machines and Turing machines were first introduced by ~\citep{Freivalds2012}. This has been followed by several papers where various aspects of them are studied in depth. ~\citep{ KasparsBalodis2013} have studied the descriptional complexity of ultrametric automata. They show that ultrametric automata can achieve an exponential advantage in terms of the number of states required when compared to equivalent deterministic automata. ~\citep{Krislauks2013} have studied the reversal complexity of ultrametric Turing machines.

Ultrametric machines are similar to probabilistic machines, the difference being that it is not necessary for amplitudes (which are the equivalent of probabilities in probabilistic automata) to be in the range between 0 and 1. Instead arbitrary rational numbers can be used. We should note that in ~\citep{Turakainen1969} a similar generalization of probabillistic automata was introduced, where "probabilities" can be arbitrarily large numbers, and the acceptance criterion is whether the probability to be in an accepting state is greater than a given threshold, furthermore it was shown that this generalization is in fact equivalent to probabilistic automata. However, unlike in these pseudo-probabilistic machines, the definition of ultrametric machines uses the concept of a $p$-adic norm.

It can be said that the definition introduced by Freivalds is natural since it uncovers the only remaining notation %TODO pārfrāzēt
of a random variable not yet fully explored ~\citep{Freivalds2012}. In addition useful properties have been proven for the definition of ultrametric machines -- ~\citep{KasparsBalodis2013} proved that the language class recognized by regulated $p$-adic machines coincides with the class of regular languages.

In this paper most of the attention is given to the question of the language hierarchy associated with the head count of ultrametric multi-head automata. %TODO pārfrāzēt
Several results can be found in the literature which consider deterministic, nondeterministic and probabilistic finite automata in both -- the two-way and one-way -- cases ~\citep{Holzer2009, Yao1978, Monien1980, Macarie1995}. The paper examines whether similar results regarding the seperation in classes with respect to the head count can be achieved for ultrametric multi-head finite automata. Other results consider the relationships of the language classes recognized by ultrametric and classical automata.